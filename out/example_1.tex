\newif\ifb % 2da clase
\newif\ifa % 3era clase...
\bfalse
\afalse

\documentclass[a4paper,11pt]{article}
\usepackage[utf8]{inputenc}
\usepackage{enumerate}
\usepackage{listings}

\title{Actividad 1 - Figuritas }

\begin{document}
	\maketitle
	\section*{?`Cu\'antas figus hay que comprar para llenar el \'album?}
	\subsection*{Simulaciones}

	\begin{enumerate}
		\item Implementar una funci\'on  que genere un \'album de tama\~no \lstinline{figus_total}, 
		simule su llenado, y devuelva la cantidad \ifb{de figuritas }\fi que se tuvieron que adquirir para completarlo.
	
		Recordar, al comienzo del \textit{script}, importar el m\'odulo \texttt{random} con el comando \textbf{import random} para poder, por ejemplo, poder simular el comportamiento de un dado numerado de 0 a 5 haciendo \lstinline{random.randint(0,5)}.
		\ifb
			\item Implementar una funci\'on que debe devolver una lista con una cantidad de elementos igual a \lstinline{n_albumes}, donde cada elemento sea la cantidad de figuritas 
		necesarias para llenar el álbum\ifa{ en cada simulación}\fi.
		\fi
	\end{enumerate}
\end{document}
